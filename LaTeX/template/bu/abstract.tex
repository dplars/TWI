Het extended abstract of de wetenschappelijke samenvatting wordt in het Engels geschreven en bevat 500 tot 1.500 woorden. Dit abstract moet niet in KU Loket opgeladen worden (vanwege de beperkte beschikbare ruimte daar).

\textbf{Keywords}: Voeg een vijftal keywords in (bv: Latex-template, thesis, lang document, ...)
