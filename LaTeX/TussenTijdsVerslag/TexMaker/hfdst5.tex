%%%%%%%%%%%%%%%%%%%%%%%%%%%%%%%%%%%%%%%%%%%%%%%%%%%%%%%%%%%%%%%%%%% 
%                                                                 %
%                            CHAPTER                              %
%                                                                 %
%%%%%%%%%%%%%%%%%%%%%%%%%%%%%%%%%%%%%%%%%%%%%%%%%%%%%%%%%%%%%%%%%%% 
\chapter{Conclusion}
\label{chap:conc}

This paper proposed a solution for the classification of tool wear on carbide inserts used in the milling industry. This will be used for the validation of camera setups in a bigger research where the research question: "How can tool wear on carbide inserts be quantized using a direct measurement method?" will be answered. This will also lead to the answer on another research question where this will validate a vision set-up.

We obtained a full dataset of white and red pictures of 140 Carbide inserts. Which have two cutting sides each so 280 images are produced.

The needed depth of the algorithm was discussed where the more complex problem with different insert types needs a deeper neural network architecture

A test accuracy of 82.1\% was achieved on the Spaghetti dataset which contained 5 different types of inserts. This result compares well to the state of the art where similar experiments where conducted with a test accuracy of 87\%. This is even outperformed with a test accuracy of 100\% on the dataset with only one type of insert.
 
\paragraph{Future work}
 This paper will be implemented to find the best method for tool wear quantisation on carbide inserts. The best set-up will be defined upon the results gathered with this research. This will be used to create a new and bigger dataset which will be used for image regression which will predict exact values for the tool wear.
 
 \paragraph{Schedule}
A renewed schedule is given in Appendix A. 
 