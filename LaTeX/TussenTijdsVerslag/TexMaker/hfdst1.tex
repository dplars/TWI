%%%%%%%%%%%%%%%%%%%%%%%%%%%%%%%%%%%%%%%%%%%%%%%%%%%%%%%%%%%%%%%%%%% 
%                                                                 %
%                            CHAPTER                              %
%                                                                 %
%%%%%%%%%%%%%%%%%%%%%%%%%%%%%%%%%%%%%%%%%%%%%%%%%%%%%%%%%%%%%%%%%%% 

\chapter{Introduction}

\section{Situreing}


\section{Problem definition}

    The steel industry is a big industry in the world and provides lots of jobs. due to high labor costs in western europe and more specific in Belgium, production and manufacturing costs should lower to be able to compete with other big companies around the world. To archive this goal we will take on one step of the manufacturing of steel which is the contol of tools used to carve in the metal.
    
    These tools typically have a base tool and a cutting plane mounted on them which is easily removed when it wears. In the past people had two options to check the wear of these planes: 
    \begin{enumerate}
        \item shutting down the whole plant and manually inspect all planes.
        \item Replacing all planes on set times.
    \end{enumerate}
     Although option number one will optimize the lifespan of the planes, it is very time consuming and labor intensive. Option number two is less labor intensive and thus less time consuming but will produce a lot of waste in those planes. Having safety levels there will be a lot of planes trown away even when they are still usable.
     
     A new way of dealing with this problem is based on option one where each plane will be inspected. To speed op this inspection process a camera will be used. And using deep neural networks the amount of wear will be projected.
     
\section{How does the tool wear?}
The wear of the tool begins with wear on the coating and goes right through the the coating in the base material of the tool. 
This base material will mostly be carbide due to its strength and heat resistance. 
	Slijtage is te zien in paper: Tool life and wear mechanism of uncoated and coated milling inserts
	Hier zijn alle slijtage types opgesomd
	
	The carbide used:
		cemented carbide here the combination with Wolfram called tungsten carbide
		in short WC Wolfram carbide
		
		more information on:
			https://www.destinytool.com/carbide-substrate.html