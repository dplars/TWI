\documentclass{scrartcl}
\usepackage[mathletters]{ucs}
\usepackage[utf8x]{inputenc}
\usepackage{amssymb}
\usepackage{amsmath}
\usepackage[usenames]{color}
\usepackage{hyperref}
\usepackage{wasysym}
\usepackage{graphicx}
\usepackage[normalem]{ulem}
\usepackage{enumerate}

\usepackage{listings}

\lstset{ %
basicstyle=\footnotesize,       % the size of the fonts that are used for the code
showspaces=false,               % show spaces adding particular underscores
showstringspaces=false,         % underline spaces within strings
showtabs=false,                 % show tabs within strings adding particular underscores
frame=single,                   % adds a frame around the code
tabsize=2,                      % sets default tabsize to 2 spaces
breaklines=true,                % sets automatic line breaking
breakatwhitespace=false,        % sets if automatic breaks should only happen at whitespace
}


\title{All Networks 1}
\date{dinsdag 08 december 2020}
\author{}

\begin{document}

\maketitle

		\section{All Networks 1}

Created woensdag 02 december 2020



All networks 1 can be found \href{https://colab.research.google.com/drive/1bWt0DgypiYoOOgnSGjlmSdDo7elvJnH4}{here.}



\subsection{Network architectures}

This model is used to test different model architectures namely:

\begin{itemize}
\item Resnet18
\item Alexnet
\item VGG11\_bn
\item Squeezenet
\item Densenet
\end{itemize}


inception v3 didn't seem to work



These models are all relatively small and should provide quite good results for a small dataset. 

More on why the models should work can be found \href{../../../../Research/Vision_Algorithm/camera_position_validation.tex}{here}



\subsection{Dataset}



This algorithm had the input of images from the \href{../../../Dataset/handmade_datasets/Second_handmade_dataset.tex}{Second handmade dataset} which was devided into 3 classes based on their measured wear value. 

\begin{tabular}{ |l|l|l| }
\hline
 class & min value (micron) & max value (micron) \tabularnewline
\hline
\hline
 good & 0 & 130 \tabularnewline
\hline
 medium & 130 & 230 \tabularnewline
\hline
 bad & 230 & ∞ \tabularnewline
\hline
\end{tabular}










\subsection{Results}

Results of this notebook are available on wandb as \href{https://wandb.ai/dplars/pytorch-TWI_second_handmade?workspace=user-dplars}{pytorch-TWI\_second\_handmade}



Interesting results will be bespoken here;

Tests for different models: 

\begin{tabular}{ |l|l|l|l| }
\hline
 model name & test accuracy \% & validation accuracy\% & transfer learning \tabularnewline
\hline
\hline
 Alexnet & 100 & 90 & yes \tabularnewline
\hline
 VGG11\_bn & 89 & 85 & yes \tabularnewline
\hline
 Densenet & 89 & 85 & yes \tabularnewline
\hline
 Squeezenet & 89 & 85 & yes \tabularnewline
\hline
 Resnet18 & 89 & 90 & no \tabularnewline
\hline
\end{tabular}
										

An overview of the best runs for every model architecture. Since there are only nine test images; the test scores are set to a very high granularity. Further results of this test are to be found on wandb as \href{https://wandb.ai/dplars/pytorch-TWI_second_handmade/reports/Testing-on-first-handmade-dataset--VmlldzozNTE5NzM}{Testing on first handmade dataset}



\end{document}
