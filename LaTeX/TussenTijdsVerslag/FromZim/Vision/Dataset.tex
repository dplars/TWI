\documentclass{scrartcl}
\usepackage[mathletters]{ucs}
\usepackage[utf8x]{inputenc}
\usepackage{amssymb}
\usepackage{amsmath}
\usepackage[usenames]{color}
\usepackage{hyperref}
\usepackage{wasysym}
\usepackage{graphicx}
\usepackage[normalem]{ulem}
\usepackage{enumerate}

\usepackage{listings}

\lstset{ %
basicstyle=\footnotesize,       % the size of the fonts that are used for the code
showspaces=false,               % show spaces adding particular underscores
showstringspaces=false,         % underline spaces within strings
showtabs=false,                 % show tabs within strings adding particular underscores
frame=single,                   % adds a frame around the code
tabsize=2,                      % sets default tabsize to 2 spaces
breaklines=true,                % sets automatic line breaking
breakatwhitespace=false,        % sets if automatic breaks should only happen at whitespace
}


\title{Dataset}
\date{dinsdag 08 december 2020}
\author{}

\begin{document}

\maketitle

		\section{Dataset}

Created vrijdag 13 november 2020



A separation is made between hand made datasets and automated datasets because they take a very different approach and produce very differing pictures.



\subsection{\href{./Dataset/handmade_datasets.tex}{Handmade datasets}}

The following datasets where produced using a microscopic camera to take pictures of single inserts all placed under the camera by hand.



\subsubsection{\href{./Dataset/handmade_datasets/initial_dataset.tex}{initial dataset}}

	The initial dataset where the images made by a microscopic camera at Sirris. These pictures were taken for the measurement of the toolwear. This dataset provided the labels for the first 5 batches labeled with 00x.
	


\subsubsection{\href{./Dataset/handmade_datasets/Second_handmade_dataset.tex}{Second handmade dataset}}

	A second dataset was made to compare the pictures with the previous dataset. This is done to verify the images and the results and to determine 	what the marker meant. This dataset also handles the first 5 batches labeled with 00x
	


\subsubsection{\href{./Dataset/handmade_datasets/second_initial_dataset.tex}{second initial dataset}}

	The second initial dataset was made with the inserts from batches 11 to 19 labeled with 01x. Here the images where taken with the same microscope as the first initial dataset but instead of phtotographing only the one insert at a time; two inserts are photographed per shot.
	


\subsection{\href{./Dataset/automated_datasets.tex}{Automated datasets}}

First the camera position is discussed and than the datasets are all listed.



\subsubsection{\href{./Dataset/automated_datasets/1_check_camera_position.tex}{Camera position}}

	This discussed two setups where on the one the camerea is more pointed to the side of the insert and the other one is pointed more to the top of the insert.
	
	\begin{enumerate}[1]
	\item \href{./Dataset/automated_datasets/1_check_camera_position/1_camera_position_side.tex}{1 camera position side dataset}
	\item \href{./Dataset/automated_datasets/1_check_camera_position/2_camera_position_top.tex}{2 camera position top dataset}
	\end{enumerate}


\subsubsection{\href{./Dataset/automated_datasets/2_created_datasets.tex}{created datasets}}

	All created datasets which conduct a few images that are worth processing are discussed here. 
	


	\begin{enumerate}[1]
	\item \href{./Dataset/automated_datasets/2_created_datasets/1_Birthday_dataset.tex}{birthday dataset}
		\begin{enumerate}[a]
		\item \href{./Dataset/automated_datasets/2_created_datasets/1_Birthday_dataset/conducted_tests_before_creation.tex}{conducted tests}
		\end{enumerate}
	\item Spaghetti dataset
		\begin{enumerate}[a]
		\item conducted tests
		\end{enumerate}
	\end{enumerate}


\end{document}
