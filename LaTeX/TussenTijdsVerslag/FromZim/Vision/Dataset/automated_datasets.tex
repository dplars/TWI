\documentclass{scrartcl}
\usepackage[mathletters]{ucs}
\usepackage[utf8x]{inputenc}
\usepackage{amssymb}
\usepackage{amsmath}
\usepackage[usenames]{color}
\usepackage{hyperref}
\usepackage{wasysym}
\usepackage{graphicx}
\usepackage[normalem]{ulem}
\usepackage{enumerate}

\usepackage{listings}

\lstset{ %
basicstyle=\footnotesize,       % the size of the fonts that are used for the code
showspaces=false,               % show spaces adding particular underscores
showstringspaces=false,         % underline spaces within strings
showtabs=false,                 % show tabs within strings adding particular underscores
frame=single,                   % adds a frame around the code
tabsize=2,                      % sets default tabsize to 2 spaces
breaklines=true,                % sets automatic line breaking
breakatwhitespace=false,        % sets if automatic breaks should only happen at whitespace
}


\title{automated datasets}
\date{dinsdag 08 december 2020}
\author{}

\begin{document}

\maketitle

		\section{automated datasets}

Created vrijdag 04 december 2020



Divided in a few topics:



\begin{itemize}
\item \href{./automated_datasets/1_check_camera_position.tex}{check camera position} which will discuss different camera position angles
\item \href{./automated_datasets/2_created_datasets.tex}{created datasets} where all fully created datasets will be discussed
\end{itemize}


\end{document}
