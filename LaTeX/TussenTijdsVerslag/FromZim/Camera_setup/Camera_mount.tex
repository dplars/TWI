\documentclass{scrartcl}
\usepackage[mathletters]{ucs}
\usepackage[utf8x]{inputenc}
\usepackage{amssymb}
\usepackage{amsmath}
\usepackage[usenames]{color}
\usepackage{hyperref}
\usepackage{wasysym}
\usepackage{graphicx}
\usepackage[normalem]{ulem}
\usepackage{enumerate}

\usepackage{listings}

\lstset{ %
basicstyle=\footnotesize,       % the size of the fonts that are used for the code
showspaces=false,               % show spaces adding particular underscores
showstringspaces=false,         % underline spaces within strings
showtabs=false,                 % show tabs within strings adding particular underscores
frame=single,                   % adds a frame around the code
tabsize=2,                      % sets default tabsize to 2 spaces
breaklines=true,                % sets automatic line breaking
breakatwhitespace=false,        % sets if automatic breaks should only happen at whitespace
}


\title{Camera mount}
\date{dinsdag 08 december 2020}
\author{}

\begin{document}

\maketitle

		\section{Camera mount}

Created Wednesday 28 October 2020



In this page the camera mount will be discussed along the design process of the setup



\begin{enumerate}[1]
\item \href{./Camera_mount/First_Camera_Mount.tex}{First Camera Mount:} a simple setup to check the results of the camera and the consistency of the images of different tools in the same \href{./Tool_Holder/Simple_holder.tex}{Holder.}
\item \href{./Camera_mount/Wheel_Camera_Mount.tex}{Wheel Camera Mount} the camera mount which is placed so that 20 tools can be photographed on one \href{./Tool_Holder/Wheel_Holder.tex}{Wheel Holder.} 
\end{enumerate}


\end{document}
