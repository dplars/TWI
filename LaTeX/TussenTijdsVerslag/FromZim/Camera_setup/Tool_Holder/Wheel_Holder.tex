\documentclass{scrartcl}
\usepackage[mathletters]{ucs}
\usepackage[utf8x]{inputenc}
\usepackage{amssymb}
\usepackage{amsmath}
\usepackage[usenames]{color}
\usepackage{hyperref}
\usepackage{wasysym}
\usepackage{graphicx}
\usepackage[normalem]{ulem}
\usepackage{enumerate}

\usepackage{listings}

\lstset{ %
basicstyle=\footnotesize,       % the size of the fonts that are used for the code
showspaces=false,               % show spaces adding particular underscores
showstringspaces=false,         % underline spaces within strings
showtabs=false,                 % show tabs within strings adding particular underscores
frame=single,                   % adds a frame around the code
tabsize=2,                      % sets default tabsize to 2 spaces
breaklines=true,                % sets automatic line breaking
breakatwhitespace=false,        % sets if automatic breaks should only happen at whitespace
}


\title{Wheel Holder}
\date{dinsdag 08 december 2020}
\author{}

\begin{document}

\maketitle

		\section{Wheel Holder}

Created Wednesday 28 October 2020



\subsection{\href{./Wheel_Holder/first_wheel_holder.tex}{First Wheel Holder}:}

A simple wheel holder which can hold 20 inserts.

used for first tests and did work. Except the tools wern't fixed good enough or they were to hard to remove and insert into the holder.

The holder was printed badly and this made the print cleanup very labor intensive. 



\subsection{\href{./Wheel_Holder/Second_Wheel_Holder.tex}{Second Wheel Holder:}}

Created on the base of the first wheel holder, but with an easier way of inserting and removing the tools. 

This with a more secure way of holding the tools. 

print so no cleanup must be done









\end{document}
