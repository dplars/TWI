\documentclass{scrartcl}
\usepackage[mathletters]{ucs}
\usepackage[utf8x]{inputenc}
\usepackage{amssymb}
\usepackage{amsmath}
\usepackage[usenames]{color}
\usepackage{hyperref}
\usepackage{wasysym}
\usepackage{graphicx}
\usepackage[normalem]{ulem}
\usepackage{enumerate}

\usepackage{listings}

\lstset{ %
basicstyle=\footnotesize,       % the size of the fonts that are used for the code
showspaces=false,               % show spaces adding particular underscores
showstringspaces=false,         % underline spaces within strings
showtabs=false,                 % show tabs within strings adding particular underscores
frame=single,                   % adds a frame around the code
tabsize=2,                      % sets default tabsize to 2 spaces
breaklines=true,                % sets automatic line breaking
breakatwhitespace=false,        % sets if automatic breaks should only happen at whitespace
}


\title{Adressable Color Changeable Led Strip}
\date{dinsdag 08 december 2020}
\author{}

\begin{document}

\maketitle

		\section{Adressable Color Changeable Led Strip}

Created Wednesday 28 October 2020



A third option of lighting is playing with the colors of the light. to archieve this a setup will be created with a single adressable light strip where the color and led can be freely chosen. 



To assign a color which works best; a study is made to find the wavelengths where the light reflects most on the used materials of the tool. This can be found \href{../../Research/Light/Light_Reflection.tex}{Light Reflection}



\end{document}
