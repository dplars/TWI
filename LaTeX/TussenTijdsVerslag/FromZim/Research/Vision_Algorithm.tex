\documentclass{scrartcl}
\usepackage[mathletters]{ucs}
\usepackage[utf8x]{inputenc}
\usepackage{amssymb}
\usepackage{amsmath}
\usepackage[usenames]{color}
\usepackage{hyperref}
\usepackage{wasysym}
\usepackage{graphicx}
\usepackage[normalem]{ulem}
\usepackage{enumerate}

\usepackage{listings}

\lstset{ %
basicstyle=\footnotesize,       % the size of the fonts that are used for the code
showspaces=false,               % show spaces adding particular underscores
showstringspaces=false,         % underline spaces within strings
showtabs=false,                 % show tabs within strings adding particular underscores
frame=single,                   % adds a frame around the code
tabsize=2,                      % sets default tabsize to 2 spaces
breaklines=true,                % sets automatic line breaking
breakatwhitespace=false,        % sets if automatic breaks should only happen at whitespace
}


\title{Vision Algorithm}
\date{dinsdag 08 december 2020}
\author{}

\begin{document}

\maketitle

		\section{Vision Algorithm}

Created woensdag 18 november 2020



\subsection{\href{./Vision_Algorithm/camera_position_validation.txt}{Finding an algorithm to test the camera position setup}}

First there must be found an algorithm that can quickly confirm wether a setup is good or not. This will be done by taking pictures different camera positions with the same lichting. After this the images go trough a simple model and the output is verified with a test set. 

This algorithm must be as small as possible to not have to take a lot of pictures to determine wether an algorithm is good or not. 







\end{document}
