\documentclass{scrartcl}
\usepackage[mathletters]{ucs}
\usepackage[utf8x]{inputenc}
\usepackage{amssymb}
\usepackage{amsmath}
\usepackage[usenames]{color}
\usepackage{hyperref}
\usepackage{wasysym}
\usepackage{graphicx}
\usepackage[normalem]{ulem}
\usepackage{enumerate}

\usepackage{listings}

\lstset{ %
basicstyle=\footnotesize,       % the size of the fonts that are used for the code
showspaces=false,               % show spaces adding particular underscores
showstringspaces=false,         % underline spaces within strings
showtabs=false,                 % show tabs within strings adding particular underscores
frame=single,                   % adds a frame around the code
tabsize=2,                      % sets default tabsize to 2 spaces
breaklines=true,                % sets automatic line breaking
breakatwhitespace=false,        % sets if automatic breaks should only happen at whitespace
}


\title{Wear}
\date{dinsdag 08 december 2020}
\author{}

\begin{document}

\maketitle

		\section{Wear}

Created woensdag 04 november 2020



The wear of the tool begins with wear on the coating and goes right through the the coating in the base material of the tool. 

This base material will mostly be carbide due to its strength and heat resistance. 

	Slijtage is te zien in paper: Tool life and wear mechanism of uncoated and coated milling inserts
	
	Hier zijn alle slijtage types opgesomd
	
	

	The carbide used:
	
		cemented carbide here the combination with Wolfram called tungsten carbide
		
		in short WC Wolfram carbide
		
		

		more information on:
		
			\href{https://www.destinytool.com/carbide-substrate.html}{https://www.destinytool.com/carbide-substrate.html}
			
			



	







\end{document}
