\documentclass{scrartcl}
\usepackage[mathletters]{ucs}
\usepackage[utf8x]{inputenc}
\usepackage{amssymb}
\usepackage{amsmath}
\usepackage[usenames]{color}
\usepackage{hyperref}
\usepackage{wasysym}
\usepackage{graphicx}
\usepackage[normalem]{ulem}
\usepackage{enumerate}

\usepackage{listings}

\lstset{ %
basicstyle=\footnotesize,       % the size of the fonts that are used for the code
showspaces=false,               % show spaces adding particular underscores
showstringspaces=false,         % underline spaces within strings
showtabs=false,                 % show tabs within strings adding particular underscores
frame=single,                   % adds a frame around the code
tabsize=2,                      % sets default tabsize to 2 spaces
breaklines=true,                % sets automatic line breaking
breakatwhitespace=false,        % sets if automatic breaks should only happen at whitespace
}


\title{Python}
\date{dinsdag 08 december 2020}
\author{}

\begin{document}

\maketitle

		\section{Python}

Created woensdag 11 november 2020



\href{./Python/installation.tex}{+installation} is done in the way described



A python script for test controlling the camera: \href{./Python/cameracontrol.tex}{+cameracontrol} 



Installation of pip \href{./Python/OpenCVinstallation.tex}{+OpenCVinstallation}



Required packages: \href{./Python/requiredPackages.tex}{+requiredPackages}



\end{document}
