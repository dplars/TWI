\documentclass{scrartcl}
\usepackage[mathletters]{ucs}
\usepackage[utf8x]{inputenc}
\usepackage{amssymb}
\usepackage{amsmath}
\usepackage[usenames]{color}
\usepackage{hyperref}
\usepackage{wasysym}
\usepackage{graphicx}
\usepackage[normalem]{ulem}
\usepackage{enumerate}

\usepackage{listings}

\lstset{ %
basicstyle=\footnotesize,       % the size of the fonts that are used for the code
showspaces=false,               % show spaces adding particular underscores
showstringspaces=false,         % underline spaces within strings
showtabs=false,                 % show tabs within strings adding particular underscores
frame=single,                   % adds a frame around the code
tabsize=2,                      % sets default tabsize to 2 spaces
breaklines=true,                % sets automatic line breaking
breakatwhitespace=false,        % sets if automatic breaks should only happen at whitespace
}


\title{motorLed}
\date{dinsdag 08 december 2020}
\author{}

\begin{document}

\maketitle

		\section{motorLed}

Created woensdag 11 november 2020



Here will the arduino code \href{./motorLed/second_test_split_strings.ino}{./second\_test\_split\_strings.ino} be discussed. Here are two addressable led strips, three long led strips and a stepper motor connected. 



the commands to be executed are the following

\begin{tabular}{ |l|l|l| }
\hline
 Commando & opties & output \tabularnewline
\hline
\hline
 LED STRIP &   &   \tabularnewline
\hline
 ------------ & -------- & -------------------------------------------------------------- \tabularnewline
\hline
 nrled & - & led off mode \tabularnewline
\hline
 nrled & + & led on mode \tabularnewline
\hline
 nrled & \_ & All leds off from set strip \tabularnewline
\hline
 nrled & xx & turn led with given index on or off depending on de set mode \tabularnewline
\hline
 nrled & +/-xx & change mode and turn led on/off \tabularnewline
\hline
   &   &   \tabularnewline
\hline
 brightness & -xx & lower brightness with xx amount \tabularnewline
\hline
 brightness & +xx & increment brightness with \tabularnewline
\hline
 brightness & \_ & set brightness to 0 \tabularnewline
\hline
 brightness & xx & set brightness to value xx \tabularnewline
\hline
   &   &   \tabularnewline
\hline
 startupleds &   & play green blue red on both ledstrips \tabularnewline
\hline
   &   &   \tabularnewline
\hline
 allon &   & turn all leds on white \tabularnewline
\hline
 alloff &   & turn all leds off \tabularnewline
\hline
   &   &   \tabularnewline
\hline
 ------------ & -------- & -------------------------------------------------------------- \tabularnewline
\hline
 Motor &   &   \tabularnewline
\hline
 ------------ & -------- & -------------------------------------------------------------- \tabularnewline
\hline
 step &   & repeat last settings for step \tabularnewline
\hline
 step & xx & repeat direction for xx steps \tabularnewline
\hline
 step & +/- & repeat number of steps in clockwise/counterclockwise direction \tabularnewline
\hline
 step & +/-xx & make xx steps in clockwise/counterclockwise direction \tabularnewline
\hline
 step & \_ & set number of steps to 0 \tabularnewline
\hline
   &   &   \tabularnewline
\hline
 q &   & turn everything off \tabularnewline
\hline
\end{tabular}


\end{document}
