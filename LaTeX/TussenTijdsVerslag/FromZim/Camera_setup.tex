\documentclass{scrartcl}
\usepackage[mathletters]{ucs}
\usepackage[utf8x]{inputenc}
\usepackage{amssymb}
\usepackage{amsmath}
\usepackage[usenames]{color}
\usepackage{hyperref}
\usepackage{wasysym}
\usepackage{graphicx}
\usepackage[normalem]{ulem}
\usepackage{enumerate}

\usepackage{listings}

\lstset{ %
basicstyle=\footnotesize,       % the size of the fonts that are used for the code
showspaces=false,               % show spaces adding particular underscores
showstringspaces=false,         % underline spaces within strings
showtabs=false,                 % show tabs within strings adding particular underscores
frame=single,                   % adds a frame around the code
tabsize=2,                      % sets default tabsize to 2 spaces
breaklines=true,                % sets automatic line breaking
breakatwhitespace=false,        % sets if automatic breaks should only happen at whitespace
}


\title{Camera setup}
\date{dinsdag 08 december 2020}
\author{}

\begin{document}

\maketitle

		\section{Camera setup}

Created zaterdag 24 oktober 2020



Tests of the setup: 

\begin{itemize}
\item 3D printing the tool holder
\item 3D printing a easy way of photographing many tools in an easy way
\end{itemize}




Parts of the setup:

\begin{enumerate}[1]
\item \href{./Camera_setup/Light.tex}{+Light}
	\begin{enumerate}[a]
	\item 3-4 led lights/strips separately controllable to light from different angles
	\item with or without extra light on top
	\end{enumerate}
\item Wheel with tool mount
	\begin{enumerate}[a]
	\item 3D model creation of mount system
	\item creating wheel to be accurate
	\item controlling stepper motor to turn just enough to put the next tool in front of the camera
	\end{enumerate}
\item Camera mount
	\begin{enumerate}[a]
	\item Design to let the camera view different angles
	\item watch out for lighting
	\end{enumerate}
\end{enumerate}






\end{document}
