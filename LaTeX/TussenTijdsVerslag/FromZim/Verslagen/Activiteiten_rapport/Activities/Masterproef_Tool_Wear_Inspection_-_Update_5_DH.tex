\documentclass{scrartcl}
\usepackage[mathletters]{ucs}
\usepackage[utf8x]{inputenc}
\usepackage{amssymb}
\usepackage{amsmath}
\usepackage[usenames]{color}
\usepackage{hyperref}
\usepackage{wasysym}
\usepackage{graphicx}
\usepackage[normalem]{ulem}
\usepackage{enumerate}

\usepackage{listings}

\lstset{ %
basicstyle=\footnotesize,       % the size of the fonts that are used for the code
showspaces=false,               % show spaces adding particular underscores
showstringspaces=false,         % underline spaces within strings
showtabs=false,                 % show tabs within strings adding particular underscores
frame=single,                   % adds a frame around the code
tabsize=2,                      % sets default tabsize to 2 spaces
breaklines=true,                % sets automatic line breaking
breakatwhitespace=false,        % sets if automatic breaks should only happen at whitespace
}


\title{Masterproef Tool Wear Inspection - Update 5 DH}
\date{dinsdag 08 december 2020}
\author{}

\begin{document}

\maketitle

		\section{Masterproef Tool Wear Inspection - Update 5 DH}

Created woensdag 25 november 2020



Meeting 25/11/2020

Dries Hulens



Te bespreken:



Wat is al gedaan?

\begin{itemize}
\item Setup bijna klaar, geen gebruik maken van de lange ledtrips, enkel de adresseerbare leds. 
\item Enkele tests met het maken van foto’s
\item Model aangemaakt Resnet18 met transfer learning om de setup mee te testen.
	\begin{itemize}
	\item Is het hier belangrijk dat er een hoog percentage gehaald wordt? 
	\item Eventueel met 60 plaatjes ook mogelijk?
	\item 3 wielen geprint waarop plaatjes geklikt kunnen worden zodat ze kunnen blijven zitten terwijl andere worden gefotografeerd.
	\item Beperkt documenteren van de keuze van de architecturen
	\item Beperkt documenteren van licht reflecties
	\end{itemize}
\end{itemize}


Wat nog gedaan moet worden dit semester:

	\begin{itemize}
	\item Heel veel schrijven aan tussentijds verslag
		\begin{itemize}
		\item Valt wel mee, goed documenteren
		\end{itemize}
	\item Setup vervolledigen met camera standaard en eventueel extra ledstrip
		\begin{itemize}
		\item Gewoon de setup nog maken en een dataset aanleggen, die dataset eens door het huidige model laten gaan mits gegevens van Tom Jacobs er zijn.
		\end{itemize}
	\item Zelf eerst foto’s filteren.
		\begin{itemize}
		\item Kijken waar een grote blob lichte pixels te zien is, die foto’s door het model laten gaan en de andere niet?
			\begin{itemize}
			\item Misschien beter dat netwerk beslist, een foto kan goed lijken voor een mens, maar het netwerk kan op andere dingen letten.
			\end{itemize}
		\end{itemize}
	\item Grotere dataset aanmaken met alle tot nu toe beschikbare plaatjes
	\end{itemize}




Vragen

	\begin{itemize}
	\item 3D printen pet PETG? 
		\begin{itemize}
		\item T hoger
		\item Bed 80°
		\item Plooibaar bed nemen
		\end{itemize}
	\item Goed als ik slechts een paar modellen implementeer en geen diepgaand onderzoek doe?
		\begin{itemize}
		\item Standaard notebook nemen waarin de verschillende naast elkaar worden getest. 
		\item Zien met suggestie van Floris De Feyter: Weights and biases -\textgreater{} is voor model optimalisatie, hoeft hier niet meteen, moet gewoon zien of iets beter is of niet
		\end{itemize}
	\item Richtlijn voor aantal pagina’s voor de related work sectie
		\begin{itemize}
		\item Afhankelijk
		\end{itemize}
	\end{itemize}


	\begin{itemize}
	\item Richtlijnen voor de gehele paper, dingen die moeten worden besproken met promotor:
		\begin{itemize}
		\item Mag we gebruikt worden in het onderzoek, aan te raden of niet?
		\item We mag zeker
		\end{itemize}
	\item Mogen de bronnen in Harvard systeem weergegeven worden?
		\begin{itemize}
		\item Harvard systeem hanteren
		\end{itemize}
	\item Wanneer moet de masterproef tekst nagelezen worden? 1 week op voorhand?
		\begin{itemize}
		\item 1 week zeker goed
		\end{itemize}
	\item Pytorch lightning?
		\begin{itemize}
		\item Mr Ophoff of de feyter eens vragen, zij gebruiken het regelmatig.
		\end{itemize}
	\end{itemize}


Cross validation

	\uline{Opzoeken} 
	
	Kan setting zijn bij dataloaders
	
	



Presentatie al eens op voorhand doorsturen naar promotor om na te laten kijken



Diepte van netwerken onderzoeken:

	Wat geeft een dieper netwerk op de verschillende aantallen foto's
	
	Darknet
	
	Resnet 
	


Beste netwerk plotten van verschillende architecturen

	om een weergave te hebben van welke netwerken het beste waren. 
	
	





\end{document}
