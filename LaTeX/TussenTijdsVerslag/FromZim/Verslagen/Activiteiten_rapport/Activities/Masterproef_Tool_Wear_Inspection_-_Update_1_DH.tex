\documentclass{scrartcl}
\usepackage[mathletters]{ucs}
\usepackage[utf8x]{inputenc}
\usepackage{amssymb}
\usepackage{amsmath}
\usepackage[usenames]{color}
\usepackage{hyperref}
\usepackage{wasysym}
\usepackage{graphicx}
\usepackage[normalem]{ulem}
\usepackage{enumerate}

\usepackage{listings}

\lstset{ %
basicstyle=\footnotesize,       % the size of the fonts that are used for the code
showspaces=false,               % show spaces adding particular underscores
showstringspaces=false,         % underline spaces within strings
showtabs=false,                 % show tabs within strings adding particular underscores
frame=single,                   % adds a frame around the code
tabsize=2,                      % sets default tabsize to 2 spaces
breaklines=true,                % sets automatic line breaking
breakatwhitespace=false,        % sets if automatic breaks should only happen at whitespace
}


\title{Masterproef Tool Wear Inspection - Update 1 DH}
\date{dinsdag 08 december 2020}
\author{}

\begin{document}

\maketitle

		\section{Masterproef Tool Wear Inspection - Update 1 DH}

Created vrijdag 20 november 2020





\subsection{Mail:}

Beste meneer Hulens,

 

Ik ben vorige week langsgegaan bij Tom Jacobs van Sirris om de eerste testplaatjes te gaan halen en heb afgesproken dat hij de volgende 200 gelabelde plaatjes kan klaarleggen tegen het einde van de maand.

Aansluitend ben ik ook gaan kijken naar de visie opstelling van Robbert bij het staalverwerkingsbedrijf Aperam waar fouten worden gedetecteerd op grote staalplaten die daar geproduceerd worden.

 

Nu ben ik bezig met het formuleren van de beschrijving van het probleem en het verleden te beschrijven. Ik vind het hier lastig om gestart te geraken en mijn weg te vinden in de hoop papers en boeken. Stilaan begin ik wel een methode te vinden om de bronnen gestructureerd te kunnen overlopen en documenteren.

 

Aan de implementatie heb ik niet meer zo heel veel gedaan. Ik heb een regressie model proberen opzetten met de Keras library, maar dat had weinig resultaat gezien Keras eigenlijk geen ondersteuning bied voor regressie problemen. Mijn volgende stap hierbij is PyTorch gebruiken en testen of dit hier wel makkelijker gaat. Keras liet ook niet toe mijn labels correct door te geven voor het classificatie model met meer dan twee klassen. Ik hoop dit ook verder uit te kunnen werken gezien Tom aangaf dat er wel degelijk een drempelwaarde van slijtage is waarbij de plaatjes moeten vervangen worden. Zo zal ik drie klassen maken: {“goed”, “bijna te vervangen”, “te vervangen”}.

Hij vertelde ook dat de plaatjes die nu gegeven worden niet door een machine gesleten worden, maar gewoon met een vijl. Hierdoor zijn de verschillende soorten slijtage niet zichtbaar. Ik het tweede semester zag hij het wel zitten om een aantal plaatjes te voorzien die wel door de machines gesleten zijn om daarop de slijtage types te herkennen.

 

Mijn planning schrijft voor om op 29/10 te starten met de uitwerking van een testopstelling, kan u tegen dan een microscoop camera voorzien en/of moet ik hiervoor verder zelf nog iets in orde brengen?

 

Ter info geef ik hier mijn notities nog mee die ik heb opgeschreven tijdens de afspraak met Tom Jacobs:

Tegen eind oktober 200 extra plaatjes

Na nieuwjaar kijken hoeveel “echte” plaatjes nodig zijn om de soorten slijtage op te kunnen bepalen

Nauwkeurigheid die gehaald zou moeten worden is 20 micrometer

Plaatje is slecht vanaf het een slijtage heeft van meer dan 200 micrometer (best iets vroeger kunnen zeggen dat het plaatje de grens van 200 micron zal overschrijden)

Hoe de inspectie nu gebeurt:

Uithalen en bekijken met blote oog of lens

Elke vaste tijdspan verwisselen

Maandelijkse update om vooruitgang te bespreken

Verschillende coatings moeten ook onderzocht worden

Kostprijs mee bekeken voor de hele setup

 

Met vriendelijke groeten,

 

Lars De Pauw



\subsection{Reply}

Beste Lars,



Zeer goed dat je de plaatjes hebt. Ik bestelde zojuist ook een nieuwe microscoopcamera waarmee je aan de slag kan.



Heeft het bezoek aan Aperam geholpen om ideeen op te doen voor jou toepassing? Misschien gebruiken ze speciale belichting ofzo om de fouten te accentueren?



Bij het beschrijven van je probleem zou ik zoeken naar technieken om abnormaliteiten te detecteren. Als je er een aantal gevonden hebt die voor voor jou interessant zijn is dat oke. Je moet er nu geen 10tallen opsommen. Wat ook interessant is zijn belichtingstechnieken om de fout duidelijker in beeld te brengen, maar ik weet niet of je daar veel over gaat vinden.



 Kan je toevallig volgende woensdag om 13u naar de campus komen? Dan ben ik daar ook en kan ik je de camera geven + even bespreken?



Mvg,



Dries Hulens



\subsection{Mail:}

Beste meneer Hulens,



Bij Aperam is de schaal wat anders en wordt een bewegende plaat gefotografeerd, dat geeft een heel verschil. Daar worden de camera en belichting op 45graden opgesteld tenopzichte van de plaat. Door de hoge fps van de camera (1000Hz) wordt een ander type belichting gebruikt om de 50Hz belichting te verbergen. Dit zal ook niet meteen van toepassing zijn bij dit onderzoek lijkt me. Ik ben wel een aantal papers tegengekomen waar men gebruikmaakt van fluorescentie lampen voor de belichtingen. Dat moet ik nog onderzoeken wat daar de voordelen van zouden zijn.



De technieken die ik tot nu toe vond zijn vooral gebaseerd op het zelf beschrijven van een model in plaats van deep learning te gebruiken. Wel was er een interessante studie die een 3D laser gebruikte in plaats van een camera waarbij zeer goede resultaten bereikt werden. 

De belichting ga ik zeker eens opzoeken wat daarvoor nodig is en wat de beste resultaten zou geven op die schaal. 



Volgende week woensdag 21/10 om 13 uur is in orde. Ik hoop tegen dan een nieuwe implementatie te kunnen tonen van het regressie model.



Met vriendelijke groeten,



Lars De Pauw



\end{document}
