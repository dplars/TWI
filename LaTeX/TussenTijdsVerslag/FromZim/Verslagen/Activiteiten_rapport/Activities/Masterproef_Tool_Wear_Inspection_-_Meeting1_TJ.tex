\documentclass{scrartcl}
\usepackage[mathletters]{ucs}
\usepackage[utf8x]{inputenc}
\usepackage{amssymb}
\usepackage{amsmath}
\usepackage[usenames]{color}
\usepackage{hyperref}
\usepackage{wasysym}
\usepackage{graphicx}
\usepackage[normalem]{ulem}
\usepackage{enumerate}

\usepackage{listings}

\lstset{ %
basicstyle=\footnotesize,       % the size of the fonts that are used for the code
showspaces=false,               % show spaces adding particular underscores
showstringspaces=false,         % underline spaces within strings
showtabs=false,                 % show tabs within strings adding particular underscores
frame=single,                   % adds a frame around the code
tabsize=2,                      % sets default tabsize to 2 spaces
breaklines=true,                % sets automatic line breaking
breakatwhitespace=false,        % sets if automatic breaks should only happen at whitespace
}


\title{Masterproef Tool Wear Inspection - Meeting1 TJ}
\date{dinsdag 08 december 2020}
\author{}

\begin{document}

\maketitle

		\section{Masterproef Tool Wear Inspection - Meeting1 TJ}

Created vrijdag 20 november 2020



Notities van de meeting

\href{./Masterproef_Tool_Wear_Inspection_-_Meeting1_TJ/meeting%2020201009.pdf}{./meeting 20201009.pdf}



Korte samenvatting:

\begin{itemize}
\item Tegen eind oktober 200 extra plaatjes
\item Na nieuwjaar kijken hoeveel “echte” plaatjes nodig zijn om de soorten slijtage op te kunnen bepalen
\item Nauwkeurigheid die gehaald zou moeten worden is 20 micrometer
\item Plaatje is slecht vanaf het een slijtage heeft van meer dan 200 micrometer (best iets vroeger kunnen zeggen dat het plaatje de grens van 200 micron zal overschrijden)
\item Hoe de inspectie nu gebeurt:
	\begin{itemize}
	\item Uithalen en bekijken met blote oog of lens
	\item Elke vaste tijdspan verwisselen
	\end{itemize}
\item Maandelijkse update om vooruitgang te bespreken
\item Verschillende coatings moeten ook onderzocht worden
\item Kostprijs mee bekeken voor de hele setup
\end{itemize}


\end{document}
