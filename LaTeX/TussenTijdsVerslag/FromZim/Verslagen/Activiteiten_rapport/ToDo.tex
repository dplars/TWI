\documentclass{scrartcl}
\usepackage[mathletters]{ucs}
\usepackage[utf8x]{inputenc}
\usepackage{amssymb}
\usepackage{amsmath}
\usepackage[usenames]{color}
\usepackage{hyperref}
\usepackage{wasysym}
\usepackage{graphicx}
\usepackage[normalem]{ulem}
\usepackage{enumerate}

\usepackage{listings}

\lstset{ %
basicstyle=\footnotesize,       % the size of the fonts that are used for the code
showspaces=false,               % show spaces adding particular underscores
showstringspaces=false,         % underline spaces within strings
showtabs=false,                 % show tabs within strings adding particular underscores
frame=single,                   % adds a frame around the code
tabsize=2,                      % sets default tabsize to 2 spaces
breaklines=true,                % sets automatic line breaking
breakatwhitespace=false,        % sets if automatic breaks should only happen at whitespace
}


\title{ToDo}
\date{dinsdag 08 december 2020}
\author{}

\begin{document}

\maketitle

		\section{ToDo}

Created vrijdag 20 november 2020



\subsection{Week 09/11/2020 - 15/11/2020}



\subsubsection{short term}

\begin{enumerate}[1]
\item netwerk maken om camera setup mee te beoordelen
\end{enumerate}


\subsubsection{Long term}

\begin{enumerate}[1]
\item Tekst schrijven
\end{enumerate}


\subsection{Week 16/11/2020 - 22/11/2020}



\subsubsection{short term}

\begin{enumerate}[1]
\item inception v3 implementeren, nieuwe foto's door het netwerk laten gaan
\item foto's nemen onder verschillende camera hoeken
\item nieuw rad printen 
\item extra clips printen in PETG
\item nieuwe camera standaard maken waarop posities te veranderen zijn.
\item Extra ledstrip toevoegen aan setup
\item Documenteren over PCB
\item Documenteren over netwerken in vloeiende tekst
\item Labels verkrijgen van extra plaatjes bij Tom Jacobs
\end{enumerate}


\subsubsection{Long term}

Grote lijnen schrijven over literatuurstudie

Mooi overzicht



\subsection{Week 23/11/2020 - 29/11/2020}

short term

\begin{enumerate}[1]
\item setup volledig afwerken 
\item dataset maken met alle plaatjes en verschillende belichtingen
\item Tussentijds verslag schrijven
\item Mail sturen Tom Jacobs
\end{enumerate}


\subsection{Week 30/11/2020 - 06/12/2020}

Verder documenteren datasets en beginnen aan documentatie van de resultaten en opschrijven wat er in het tussentijds verslag komt.



\begin{itemize}
\item[\CheckedBox] second wheel holder \href{../../Camera_setup/Tool_Holder/Wheel_Holder/Second_Wheel_Holder.tex}{Camera setup:Tool Holder:Wheel Holder:Second Wheel Holder}
\item[\CheckedBox] second initial dataset description
\item[\CheckedBox] AllNetworks 1
\item[\CheckedBox] allnetworks 2
\item[\CheckedBox] allnetworks 3
\item[\CheckedBox] all networks 4
\item[\CheckedBox] Spaghetti dataset
\item[\Square] create overview of different dataset and compare them with results
\end{itemize}


\end{document}
