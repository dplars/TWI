\documentclass{scrartcl}
\usepackage[mathletters]{ucs}
\usepackage[utf8x]{inputenc}
\usepackage{amssymb}
\usepackage{amsmath}
\usepackage[usenames]{color}
\usepackage{hyperref}
\usepackage{wasysym}
\usepackage{graphicx}
\usepackage[normalem]{ulem}
\usepackage{enumerate}

\usepackage{listings}

\lstset{ %
basicstyle=\footnotesize,       % the size of the fonts that are used for the code
showspaces=false,               % show spaces adding particular underscores
showstringspaces=false,         % underline spaces within strings
showtabs=false,                 % show tabs within strings adding particular underscores
frame=single,                   % adds a frame around the code
tabsize=2,                      % sets default tabsize to 2 spaces
breaklines=true,                % sets automatic line breaking
breakatwhitespace=false,        % sets if automatic breaks should only happen at whitespace
}


\title{Activiteiten rapport 1}
\date{dinsdag 08 december 2020}
\author{}

\begin{document}

\maketitle

		\section{Activiteiten rapport 1}

Created vrijdag 20 november 2020



Alle mails gelinkt \href{./Activities.tex}{hier}



\subsection{Wat is er reeds gedaan deze periode tot 10 November?}



\subsection{Week 05/10/2020 - 11/10/2020}

opstellen en indienen van het tussentijds verslag in de eerste helft van de week.

Na 07/10/2020 was het verslag ingediend en werd gestart met het formuleren van een beschrijving van het probleem.

Het schrijven ging zeer moeizaam omdat ik weinig perspectief had over wat het doel van de masterproef was. Het is ook zeer moeilijk om een weg te banen tussen alle literatuur en de goede er tussenuit te vinden. Er werd heel wat tijd verspild met het lezen van bronnen die onvoldoende zijn of net niet goed genoeg waren. Hier moest een systeem in gevonden worden om de bronnen snel te kunnen inschatten. 



\begin{enumerate}[1]
\item bezoek gebracht aan Tom Jacobs van Sirris om de eerste plaatjes op te halen. 
	\begin{itemize}
	\item frees getoond en 
	\item uitleg gegeven over het bredere kader van het onderzoek. 
	\item Een verslag is \href{./Activities/Masterproef_Tool_Wear_Inspection_-_Meeting1_TJ.tex}{hier} te vinden. 
	\end{itemize}
\item Bedrijfs bezoek bij Aperam waar een vriend een visie inspectie systeem heeft geïnstalleerd dat fouten detecteerd in grote stalen platen. 
	\begin{itemize}
	\item Hier was het zeer nuttig om eens te zien hoe een systeem wordt geïmplementeerd in een productielijn. 
	\item De schaal was hier wel een stuk verschillend waardoor er niet echt dingen kunnen overgenomen worden.
	\end{itemize}
\item Start aan een beschrijving van het verleden van dit onderwerp.
	\begin{itemize}
	\item ging zeer moeizaam om dezelfde redenen als het zoeken van de probleem beschrijving.
	\end{itemize}
\item Implementatie maken voor een binair transfer learning model dat bepaald of een tool versleten is of niet. te vinden in \href{https://colab.research.google.com/drive/1N2r6nplmx88pUOKkLN0Hmy42D7YRUypC}{TWI 5} Dit bouwde verder op \href{https://colab.research.google.com/drive/1Qsoj7WmGClsslivGGI4DGL10NNVdIHC2}{TWI 4} waar een zelfde model werd gemaakt zonder tranfer learning.  \href{https://colab.research.google.com/drive/1Qsoj7WmGClsslivGGI4DGL10NNVdIHC2}{TWI 4} is een volledig werkend prototype van een algoritme in Keras.
\item implementatie maken voor een classificatie model \href{https://colab.research.google.com/drive/1QN68qaE84fq9dBnZFNExsS5F8tZk-8Ow}{TWI 6} in Keras waarbij tranfer learning wordt gebruikt. Hierbij waren alle voorspellingen voor dezelfde klasse. Niet verklaard hoe dit kon, misschien omdat de data niet gelijk verdeeld was over de klassen. De beste learning rates waren reeds bepaald voor dit probleem en lagen rond 3e-5 
\end{enumerate}


Voor alle bovenstaande tests werden de labels van \href{file:///Users/larsdepauw/Documents/Lars.nosync/Documents/School/1Ma%20ing/Masterproef/TWI/code/Vision/Datasets/labels/labels3.csv}{labels3.csv} gebruikt op de \href{file:///Users/larsdepauw/Documents/Lars.nosync/Documents/School/1Ma%20ing/Masterproef/TWI/code/Vision/Datasets/handmade/first_handmade}{first handmade dataset}





\subsection{Week 12/10/2020 - 18/10/2020}



\begin{enumerate}[1]
\item Verder schrijven aan het verleden van het onderwerp. 
	\begin{enumerate}[a]
	\item Ging zeer moeizaam en vergde veel tijd voor zeer weinig resultaat. 
	\item Gekomen tot een aantal bronnen. 
	\item opzetten mendeley om bronnen in bij te kunnen houden.
	\end{enumerate}
\item Aanmaken van Engelse latex file op overleaf
\item implementatie maken in keras waarbij een regressie model wordt getraind op de eerste honderd foto's die verkregen zijn
	\begin{enumerate}[a]
	\item fotos te vinden als \href{file:///Users/larsdepauw/Documents/Lars.nosync/Documents/School/1Ma%20ing/Masterproef/TWI/code/Vision/Datasets/handmade/first_handmade}{first handmade dataset}
	\item Een regressie model wordt niet ondersteund door Keras waardoor het zeer omslachtig was om de labels en de foto's met elkaar te kunnen verbinden. Dit zorgde voor heel wat problemen.
		\begin{enumerate}[1]
		\item classificatie model met transfer learning maken met Keras dat de gegevens opdeeld in 3 klassen waarbij ook een verschillende batch size werd getest. Te zien in \href{https://colab.research.google.com/drive/1QN68qaE84fq9dBnZFNExsS5F8tZk-8Ow}{TWI 6} en  \href{https://colab.research.google.com/drive/1zeKin3shLk_ogrFCwHrzpnjJmaWaoCq0#scrollTo=CDnGBD0NGkbM}{TWI 7}
		\item proberen de labels toch gelinkt te krijgen via de data loader functies van Keras. Dit werkte echter niet dus werd een andere methode geprobeerd
		\item Het regressie probleem omzetten in een classificatie model wat betreft de data. Hiervoor werd een map aangemaakt per waarde en dus per foto om zo de gegevens in te lezen. Dan werd de model architectuur aangepast zodat deze een output gaf van een getal waarde in dezelfde grootte orde als de waarden van de metingen. 
		\end{enumerate}
	\end{enumerate}
\end{enumerate}
			Dit werkte echter ook niet en er werden geen resultaten bekomen.
			
			De uitwerking is te vinden op Google Colab onder de naam \href{https://colab.research.google.com/drive/1qR4reWAIvIs59eZKWSmZHboZ1gjGA4jS}{TWI 8} waarbij ook transfer learning werd toegepast om de weinige data toch te kunnen omzetten in een werkend model
			
\begin{enumerate}[1]
\setcounter{enumi}{3}
\item Er werd gekeken welke andere frameworks er nog zijn en wat het verschil is tussen deze. Hier is ook beslist om verder te gaan met Pytorch gezien dit het meest open framework is. 
\end{enumerate}


\subsection{Week 19/10/2020 - 25/10/2020}



\begin{enumerate}[1]
\item Een nieuwe implementatie \href{https://colab.research.google.com/drive/1f-Rfei8QrHotvgktp8hzviR8dv1XJJjz}{TWI 9} waarin een eerste model werd gemaakt met pytorch. 
	\begin{enumerate}[a]
	\item er werden wat problemen gevonden bij het tonen van de foto's eens ze genormaliseerd waren waardoor het moeilijk was om de data voor te stellen die gebruikt werd. 
	\item Een model van het internet werd geïmplementeerd en er werden resultaten bekomen ± 60\% accuracy op de train en validation data. Dat lijkt niet zo correct. op een test set van slechts enkele beelden werd 25\% accuracy gehaald. Er leek ergens iets mis te zijn met de transformaties van de foto's naar pil voorstelling. 
	\end{enumerate}
\item Een binaire classificatie implementatie \href{https://colab.research.google.com/drive/1t4Yvzj1rqNy24pj4dDtyBH2CStDePXMw}{TWI 10} werd gemaakt waarin een zelfde model architectuur werd geprobeerd te maken als bij het model van Keras. 
	\begin{enumerate}[a]
	\item Echter was dit zeer lastig zonder voorkennis van hoe architecturen er uit zien. Door het sterk verschil in benamingen tussen functies is dit niet geslaagd. 
	\item De foto's in juiste mappen plaatsen via een programma werd ook gerealiseerd na heel wat problemen met de google colab mappen structuur. 
	\item In het model waren de data augmentation regels te hard. Hierdoor werden de foto's zeer sterk bewerkt wat zorgde voor heel wat meer foto's, maar tegelijk deed het afbreuk aan het model gezien de foto's in het echt makkelijker te interpreteren vallen. 
	\item De training en validation loss ging niet naar beneden tijdens de training.
	\end{enumerate}
\item Meeting met Dries Hulens waarin werd overlopen hoe de opstelling zou gemaakt worden en waar op te letten.
	\begin{enumerate}[a]
	\item hierbij werd aangehaald dat het schrijven nog niet belangrijk zou zijn, en dat dit nog even kon wachten tot december eventueel.
	\item licht reflecties onderzoeken hoe het materiaal reageerd op bepaalde licht frequenties.
	\item eventueel kijken of een stuk uit de fotos kan gesneden worden (felste blob) om enkel daar het model mee te voeden.
	\item adresseerbare ledstrip mee gekregen
	\item camera mee gekregen.
	\end{enumerate}
\item Een eerste testopstelling maken om de camera te leren gebruiken.
	\begin{enumerate}[a]
	\item De camera moest zeer goed ingesteld worden om een scherp beeld te krijgen van de slijtage
	\item Setup is besproken in: \href{../../Camera_setup/Camera_mount/First_Camera_Mount.tex}{first camera mount} 
		\begin{enumerate}[1]
		\item hiervoor is een \href{../../Camera_setup/Tool_Holder/Simple_holder.tex}{houder} ge 3D print.
		\item De belichting besproken in \href{../../Camera_setup/Light/Desk_Lamp_Test.tex}{desk lamp test}
		\item de camera werd met de \href{../../Camera_setup/Camera_mount/First_Camera_Mount.tex}{normale houder} op de bureau gezet. 
		\end{enumerate}
	\item De beelden hiervan waren zeer indrukwekkend en gaven een mooi resultaat zonder veel werk te hoeven steken in een setup.
	\end{enumerate}
\end{enumerate}


\subsection{Week 26/10/2020 - 1/11/2020}



\begin{enumerate}[1]
\item 3D model gemaakt van het \href{../../Camera_setup/Tool_Holder/Wheel_Holder/first_wheel_holder.tex}{eerste rad} dat gebruikt zal worden voor een automatisch systeem dat de plaatjes voor de camera beweegt.
	\begin{enumerate}[a]
	\item Creatie van een 3D model
	\item 3D printen 
	\item berekening welke nauwkeurigheid nodig is om de stappen motor aan te sturen
	\end{enumerate}
\item Een programma aangemaakt om de stepper motor aan te sturen met de motor drivers. 
\item documenteren van de afgelopen tests
\item Test met vaste led strips om ze te kunnen aansturen vanaf een raspberry pi
	\begin{enumerate}[a]
	\item Ging niet goed met zeer kleine transistors. 
	\item overgeschakeld naar relais. Gaf zeer veel problemen met solid state relais die ook in gesloten toestand nog een stroom doorlieten.
	\item Gewone analoge stuurbare relais werkten wel, maar maken een klik geluid. 
	\item Volgende stap is om mosfets te gebruiken voor deze sturing.
	\item Test met verschillende voltages met een potentiometer voor de strips.
	\end{enumerate}
\item PCB maken om de aansturing mogelijk te maken gezien het breadboard de hoge stroom niet aan kan. 
\end{enumerate}


\subsection{Week 02/11/2020 - 08/11/2020}



\begin{enumerate}[1]
\item Ophalen van een volgende batch plaatjes
	\begin{enumerate}[a]
	\item bekijken hoe de plaatjes worden gelabeld. 
	\item opzoeken en documenteren of de plaatjes reageren op een bepaalde golflengte zonder goede resultaten, dit moet nog verder onderzocht worden.
	\item materialen van de plaatjes opzoeken
	\item Afslijting komt recht op de hoek van het plaatje
	\item Frees kan niet zelf nauwkeurig draaien, een mogelijke oplossing is de voledige snijplaat houder uit de frees halen en fotograferen.
	\item Er zijn ongeveer 4 hoofdklassen van slijtages, maar deze zijn moeilijk zelf na te maken
	\item Een extra metric toevoegen die de oppervlakte van de slijtage weergeeft. 
	\end{enumerate}
\item Een testopstelling gemaakt met het rad met de stepper motor en enkele bandijzers om de leds aan te bevestigen.
\item Arduino finetunen zodat deze volledig werkt met de adressable led en de motor makkelijk kan aangestuurd worden met commandos via de seriele input.
\item Eerste beelden maken met een geautomatiseerde setup
	\begin{enumerate}[a]
	\item \href{file:///Users/larsdepauw/Documents/Lars.nosync/Documents/School/1Ma%20ing/Masterproef/TWI/code/Vision/Datasets/First_automated/camera_zij_boven_dual_ledstrip/first_test}{zie first automated dataset}
	\item werkte niet, de communicatie tussen de computer en arduino was te traag. Dit wordt (tijdelijk) opgelost door delays te plaatsen die de trage communicatie opvangen. 
	\item Eerste beelden konden gemaakt worden en het python script is gemaakt.
	\end{enumerate}
\item PCB finetunen 
	\begin{enumerate}[a]
	\item nieuwe weerstand die de spanning begrensd voor de ledstrip
		\begin{enumerate}[1]
		\item is doorgebrand en terug vervangen door een grotere weerstand
		\end{enumerate}
	\end{enumerate}
\end{enumerate}




\subsection{Wat er nog zal gebeuren voor de tussentijdse presentatie}

\begin{enumerate}[1]
\item Afwerken van de setup en oplossen 
\end{enumerate}






\end{document}
